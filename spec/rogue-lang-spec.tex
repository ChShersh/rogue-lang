\documentclass[a4paper]{article}

%% Language and font encodings
\usepackage{cmap}					% поиск в PDF
\usepackage[T2A]{fontenc}			% кодировка
\usepackage[utf8]{inputenc}			% кодировка исходного текста
\usepackage[english]{babel}	% локализация и переносы
\usepackage{amsmath}
\usepackage{graphicx}
\usepackage{listings}
\usepackage{bm}
\usepackage{color}
\usepackage{xcolor}
\usepackage[colorinlistoftodos]{todonotes}
\usepackage[colorlinks=true, allcolors=blue]{hyperref}

\setlength\parindent{0pt}

%\definecolor{darkOrange}{RGB}{255, 129, 0}
\definecolor{lightGray}{RGB}{236, 236, 236}
\definecolor{myGreen}{rgb}{0,0.6,0}
\definecolor{darkBlue}{RGB}{27,0,116}

\usepackage{caption}
\DeclareCaptionFont{white}{\color{white}}
\DeclareCaptionFormat{listing}{\colorbox{gray}{\parbox{\textwidth}{#1#2#3}}}
\captionsetup[lstlisting]{format=listing,labelfont=white,textfont=white}

\renewcommand{\lstlistingname}{Listing}
\lstset{%
    backgroundcolor=\color{lightGray},
    commentstyle=\itshape\color{myGreen},
    extendedchars=true,
    keywordstyle=\bfseries\color{darkBlue},
    language=Java,
    otherkeywords={let,mut,pure,dirty,then,else,Int,Double,Bool,true,false,skip},
    numbers=left, % where to put the line-numbers; possible values are (none, left, right)
    numbersep=5pt
    %stringstyle=\color{mymauve}
}

%% Sets page size and margins
\usepackage[a4paper,top=3cm,bottom=2cm,left=3cm,right=3cm,marginparwidth=1.75cm]{geometry}

%% Spec title
\title{\textbf{Rogue} programming languages specs}
\author{Dmitry Kovanikov}
\date{}

\begin{document}
\maketitle

\textbf{Implementation language:} Haskell\\

\textbf{Target platform:} LLVM\\

\section*{Language features}

\textbf{Rogue} programming language is mix of imperative and functional paradigms. You can write algorithms step by step. But language has plenty of functional features. It's statically typed with local type inference. 

As a general purpose programming language it has commonly used set of basics:

\begin{enumerate}
\item Variable declaration.
\item Basic arithmetic, logic and comparison operations.
\item Function declaration and calling.
\item Primitive types: Int, Bool, Double, Unit (Word? String? BigInteger?)
\end{enumerate}

Imperative language features are:

\begin{enumerate}
\item Variables can be mutable and immutable.
\item Named function arguments with default values.
\item Control-flow constructs: if-then-else, while and for loops.
\item TODO: Arrays with size in type?
\end{enumerate}

Functional language features are:

\begin{enumerate}
\item Pattern matching on constants.
\item Higher-order functions.
\item Anonymous functions.
\item Currying and partial application.
\item Classification of functions on pure and with side-effects.
\item Algebraic immutable data types.
\item TODO: parametric polymorphism.
\end{enumerate}

TODO: fully-compatible with Haskell.

\section*{Syntax examples}

\begin{lstlisting}[caption=Variables declaration]
mut x: Int = 0
mut y = 3
let z = true   // `z` has type Bool
y = 5
\end{lstlisting}

Using \textbf{mut} keyword you can create mutable variables, and with \textbf{let} --- immutable. Note, that you can omit type of variable, if it can be inferred from it's local context. But variable initialization is required.

\begin{lstlisting}[caption=Function declaration]
pure f : (x: Double) -> (y: Double) -> Double {
    let z = x + y
    return z
}

dirty 
g : (b: Bool) -> (mut x: Int) -> (y: Int = 5) -> Int 
g (..) = {
    if b { x += y }
    return x
}
\end{lstlisting}

Arguments of functions separated with $\rightarrow$ symbol. They should have names and type. Arguments can have default values. Last type doesn't have name "--- it is the type of function result. All functions should be marked either \textbf{pure} or \textbf{dirty}. If function has side-effects (changes some of it's arguments or some global variables or does some IO) then it should be marked as \textbf{dirty}. Function block should have \textbf{return} keyword if result type of function is not \textbf{Unit}.

Next calls of function \texttt{g} are valid:

\begin{lstlisting}[caption=Function calls]
g true someX 4 
g true someX    // creates function with type `Int -> Int`
g(b = true, x = someX)  // calls `g` with `y = 5`
g(x = someX)  // creates function with type Bool -> Int
g() // creates function with type `Bool -> mut Int -> Int`
\end{lstlisting}

So \texttt{g \{x = someX\} true} is valid call whereas \texttt{g true \{x = someX\}} is not
because partial function application loses all information about arguments names. Though by convention you shouldn't write \texttt{g \{x = someX\} true} because it's less obvious what this function call does.

TODO: Call like \texttt{g(..) }

Functions also can perform pattern matching, have guards, etc.

\begin{lstlisting}[caption=Pattern matching]
dirty h : (b: Bool) -> (mut x: Int) -> (y: Int) -> Int
h (y = 0) = {
    x = x - 3
    let a = x / 2
    return x + a
}
h (y = 1) 
    | x > 1 = x - 1
    | else  = { let t = x % 2; return t }
h false 1 _ = x + 10  // `y` is not available here
h true (..) = y + 10  // {..} for keeping rest arguments names
h (..) = x + 1
\end{lstlisting}

Here's example of function that reads two integers from input and performs binary pow aglorithm.

\begin{lstlisting}[caption=Binary pow algorithm]
dirty binPow : Unit {
    mut k = readInt()
    mut n = readInt()
    mut res = 1
    while k > 0 {
        if k % 2 == 1 {  
            res *= n
        } else {
            skip // `skip` is empty operator
        }
        n = n * n
        k = k / 2
    }
    print res
}
\end{lstlisting}

Example of calling with higher-order functions.

\begin{lstlisting}[caption=Higher order functions]
dirty decrementAndCheck : (mut x: Int) 
                       -> (p: mut Int -> Int -> Bool) 
                       -> Bool 
{
    x = x - 1
    return (p x 3)  
}
\end{lstlisting}
\end{document}